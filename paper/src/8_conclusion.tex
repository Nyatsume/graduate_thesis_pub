\chapter{結論}
\label{conclusion}

本章では,本研究のまとめと今後の課題を示す.

\section{本研究のまとめ}
本研究では,IPv6シングルスタックネットワークにおけるIPv4サービス提供手法として,ステートレスアドレス変換を利用した"SIIT-DC"と呼ばれるネットワークデザインに注目し,冗長性や変更追従性の問題を解決するために,考えられるアプローチを比較した上で,動的経路制御プロトコルであるBGPを利用したアドレス変換テーブルの広告・更新プロトコルを設計した.

本手法を評価するために,OSS及び自作のを利用したPoCを実装し,30台のBR,2台のルートリフレクタ,最大600台のサーバからなる評価用ネットワークにて2つのシナリオからなる評価実験を行った.
本評価実験の結果,本手法がSIIT-DCの冗長性と変更追従性を改善するフィジビリティを十分に有することが明らかになった.

\section{今後の展望}
\subsection{本提案手法に適したBGPコネクショントポロジについての検討}
第\ref{evaluation:consideration}節で述べたように,ルートリフレクタが保有するべきBGPコネクションが大きくなることが,パフォーマンスに影響を及ぼすことがわかっている.
今後本提案手法のパフォーマンスをより一層高めるために,本提案手法のメリットを活かしながらも,ルートリフレクタの冗長性負荷軽減につながるようなBGPコネクショントポロジを設計・提案していく必要がある.


\subsection{EAMの集約}
本提案手法では,IPv4サービス提供サーバ自身がIPv4サービスアドレスを広告することを前提とした設計を行った.
IPv4サービスアドレスを有するサーバ以外がEAMを複数集約して代理に広告するというユースケースを想定すると,BGPが利用するアドレスファミリの拡張仕様の設計が必要になると言える.


%%% Local Variables:
%%% mode: japanese-latex
%%% TeX-master: "../thesis"
%%% End:
