\chapter{プロトコル設計と実装}
\label{implementation}
本章では,第\ref{proposal}章で述べた提案システムのメッセージ設計と実装について述べる.



\section{BGPメッセージの設計}
\subsection{要件}
Dynamic EAMTに必要な属性は以下の3種
\begin{itemize}
    \item IPv4サービスアドレス \\
    \item IPv6サービスアドレス \\
    \item 変換プレフィックス \\
\end{itemize}

\subsection{実装}
\subsubsection{アドレスファミリー}
IPv6 ユニキャスト経路で実現
\subsubsection{メッセージフォーマット}

BGP UPDATE Messageのパケットを図で表す。

\begin{itemize}
    \item reachable\_NLRI \\
    \item unreachable\_NLRI \\
    \item NEXTHOP \\
\end{itemize}


\section{PoCの実装}

\subsection{各コンポーネントの実装}
refで述べたコンポーネントは以下の様にそれぞれ実装した.
\subsubsection{BGPデーモン}
BGPデーモンには,OSSのBGPデーモンであるGoBGPを利用する.gRPCでサービス連携が楽である.
\subsubsection{SIIT}
SIIT機構にはJoolを使う.Linux Netfilterで実装されていてどこでも持ってける.
\subsection{EAMT制御機構}
BGP Loc-RIBをwatchしてSIIT機構の変換テーブルを書き換える仕事をする.
EAMT制御機構には手作りPythonスクリプトを使う
表で使用ライブラリなどを書く.


%%% Local Variables:
%%% mode: japanese-latex
%%% TeX-master: "./thesis"
%%% End:
