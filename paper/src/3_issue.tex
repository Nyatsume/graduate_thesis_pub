\chapter{SIIT-DCのデザインと現状の課題}
\label{issue}
第\ref{related:compare:translation}で述べたPv4/IPv6トランスレーションを用いたIPv4サービス提供手法の一つとして,SIIT-DCがインターネット標準化されている.
本章ではてSIIT-DCのデザインとメリット及び考えられる運用,そして現状の課題について述べる.
\section{SIIT-DC}

\subsection{概要}
SIIT-DCとは,ステートレスIP/ICMP変換アルゴリズム\cite{RFC7915}を利用して,IPv4インターネット・ネットワークからのアクセスをIPv6シングルスタックネットワーク上のホストに提供するためのネットワークデザインである.2016年よりIETF IPv6 Operations WG\footnote{IPv6ネットワークの運用要件や関連する技術仕様の策定を行うワーキンググループ.\url{https://datatracker.ietf.org/wg/v6ops/about/}}によりインターネット標準化(Informational RFC)されている\cite{RFC7755}.

SIIT-DCの基本的なアーキテクチャを図(書く)に示す.


\subsection{用語}
SIIT-DCで利用される用語及び特殊な役割を有する機器・技術について述べる.

\subsubsection{SIIT}
SIIT\footnote{Stateless IP/ICMP Translation Algorithm}とはIPv4/IPv6トランスレーションに用いられるプロトコル変換機能の略称である.RFC2765\cite{RFC2765}で初めて標準化され,その後RFC6145\cite{RFC6145}により一部の仕様が実運用のユースケースに合わせて変更され,現在はIPv6拡張ヘッダーを扱う機構などが追加されたRFC7915\cite{RFC7915}が現行の標準仕様である.

\subsubsection{BR}
BR\footnote{Border Relay}とは,SIIT-DCネットワークにおいてIPv4インターネットとIPv6ネットワークとの間でSIITによるIPv4/IPv6トランスレーションを行う機器\footnote{専用機器もしくは他の役割を有する機器の一機構.}である.
IPv4インターネットとIDC内のIPv6シングルスタックネットワークの各境界部に所在し,後述するEAMTを参照した1:1のアドレス変換を行う.IDCネットワークにIPv4インターネットとの接続点が複数ある場合,接続点ごとに最低一つのBRを配備する.

\subsubsection{ER}
ER\footnote{Edge Relay}とは,IPv4ネットワークとIPv6ネットワークとの境界点において多:多のIPv4/IPv6トランスレーションを行う機器である.
SIIT-DCではそのオプションとして,IDCネットワーク内のIPv6ネットワークとPv4しか利用出来ないホストが属するIPv4ネットワークとの間を接続するユースケースをサポートするSIIT-DC Dual Translation Mode\cite{RFC7756}が定義されており,ERはその中での利用が想定されている.

\subsubsection{EAM}

\subsubsection{IPv4サービスアドレス}

\subsubsection{IPv6サービスアドレス}

\subsubsection{Translation Prefix}
本研究ではTranslation Prefixを変換プレフィックスと呼称する.




\subsection{BRの動作}


\subsection{ネットワーク設計}



\section{SIIT-DCの問題}

とにかくEAMTがダイナミックじゃないことに起因すると言う
変更追従性の欠如
一貫性の欠如


%%% Local Variables:
%%% mode: japanese-latex
%%% TeX-master: "./thesis"
%%% End:
