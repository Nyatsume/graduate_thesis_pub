\chapter{本提案手法の設計}
\label{proposal}
第\ref{consideration}章では,DynamicEAMTを設計する上で考えられる二種類のアプローチについて,求められる要件に照らし合わせて評価・検討を行った.
本章では検討結果の得られた内容を基に,iBGPを利用したDynamicEAMTの実現手法の設計に関して論じる.


本章では提案手法の設計を述べる.

\section{概要}
分散管理と中央管理のいいところどりが可能な、iBGP + RR構成を取る


\section{BGP}
\subsection{概要}
用途

iBGPとEBGP

\subsection{特徴}
\subsubsection{マルチプロトコル}
\subsubsection{実装が一般的}
OSSもある
\subsubsection{軽量}

\subsection{iBGPとルートリフレクター}
\subsubsection{用途}
\subsubsection{Dynamic EAMTでのメリット}

\section{ネットワーク設計}
ネットワーク図を書く

\section{各ノードの役割と要件}
\subsection{BR}

\subsubsection{BGPデーモン}
\subsection{SIIT機構}
\subsection{EAMT制御機構}



\subsection{IPv4サービス提供サーバー}
\subsubsection{BGPデーモン}
\subsubsection{サービスアドレス広告機構}


\subsection{RR}
\subsubsection{BGPデーモン}
\subsubsection{ルートリフレクター機構}

\subsection{各アプローチとの比較}
星取り表を書く

%%% Local Variables:
%%% mode: japanese-latex
%%% TeX-master: "../bthesis"
%%% End:
