修士論文要旨 - 2019年度 (令和元年度)
\begin{center}
\begin{large}
\begin{tabular}{|M{0.97\linewidth}|}
    \hline
      \title \\
    \hline
\end{tabular}
\end{large}
\end{center}

現在インターネットは,IPv4を用いたネットワークからIPv6の新しいインターネットへの移行が加速している.ISP(Internet Service Provider)の多くは,IPv4/IPv6デュアルスタック運用をおこなうことにより,すでにIPv6の利用環境が整備され,PCに代表される端末機器でもIPv6の利用が前提となってきている.このような中,インターネットを介したサービスを提供するコンテンツ事業者やIDC(Internet Data Center)におけるIPv6の対応が,大きな課題となってきている.コンテンツ事業者やIDCでは,サービス数や膨大な数のサーバを運用する必要があり,IPv4アドレスが枯渇する中,Dual Stack運用が難しく,IPv6シングルスタックでの運用が期待されている.しかし,既存のユーザの多くは依然IPv4ネットワークを利用していることから,IPv6シングルスタックでの運用であっても,IPv4によるサービスを継続して提供可能なネットワーク設計が求められている.

IPv6シングルスタックネットワークにおけるIPv4によるサービス提供技術として,SIIT-DCと呼ばれる手法がIETFにおいて標準化されている.SIIT-DCはBR(Border Relay)と呼ばれる変換機構をIPv4インターネットとIDC内のIPv6ネットワークの境界点に設置し,アドレス変換テーブル(EAMT:Explicit Address Mapping Table)を参照してプロトコル変換を行うことで,IPv6サーバでのIPv4サービス提供を可能にする技術である.しかしながらSIIT-DCではEAMTの動的な制御方法についての定義がされておらず,複数のBRを利用するような環境でのEAMTの一貫性の確保や,IPv4でサービス提供を行なうサーバの構成変更時のEAMTの設定変更などに課題がある.SIIT-DCのこれらの問題を解決するためには,BRが保有するEAMTを動的に制御する”ダイナミックEAMT機構”が必要である.

本研究ではBGP(Border Gateway Protocol)を利用したダイナミックEAMT機構を提案し,現在多くのサービスを提供しているコンテンツ事業者やIDCにおいてIPv6シングルスタック運用を可能とした.BGPはインターネットで広く利用されている動的経路制御プロトコルであり,経路情報を一貫性のある形で共有することが出来るほか,BGPピアの変更に対して追従する機構を有している.本提案手法ではIBGP(Internal BGP)をEAMT管理機構に適応するための機能拡張を行うと共に,ルートリフレクタを利用することによりスケーラブルなダイナミックEAMT管理制御機構を実現する.

本手法を評価するために,BGPによるEAMT管理制御機構を実装したソフトウェアルータを実装し,IDCネットワークを模した概念実証用ネットワーク上で評価実験を行った.本評価環境においては,本提案手法が1階層のルートリフレクタ構成で,最大600台のIPv4サービス提供サーバの情報から,30台のBRのEAMTを一貫性のある形でダイナミックに制御可能なことを明らかにした.SIIT-DCのEAMTの一貫性の維持や変更追従性の面で,本提案手法が効果的に作用することが証明できた.また,多層のルートリフレクタ構成を用いることにより実際にコンテンツ事業者などが運用する規模でのサービス提供が可能なシステム設計もおこなった.

本研究で提案した手法は,IPv6シングルスタックネットワークにおけるIPv4サービス提供の質向上を実現可能であり,今後のIDCネットワーク設計に貢献することが期待される.



~ \\

キーワード:\\
\underline{1. データセンターネットワーク},
\underline{2. ネットワークオペレーション},
\underline{3. IPv6移行技術}
\begin{flushright}
\dept \\
\author
\end{flushright}
