学士論文要旨 - 2020年度 (令和2年度)
\begin{center}
\begin{large}
\begin{tabular}{|M{0.97\linewidth}|}
    \hline
      \title \\
    \hline
\end{tabular}
\end{large}
\end{center}

近年,ロードバランサーはハードウェアアプライアンスのものからソフトウェアのものへ変遷しつつある.これにより,データセンターは物理スペースや経済的なコスト問題を解決した.これらの高速ソフトウェアロードバランサにはカーネルバイパス系パケット処理フレームワーク等を利用した高速データプレーン技術を用いたパケット処理性能の向上が図られているが,これらの高速データプレーン技術は既存のコントロールプレーンのAPIを利用することができず,開発者はデータプレーンよりもコントロールプレーンの開発にコストを掛けているのが実情である.

そこで本研究では,既存のコントロールプレーン機能を流用しつつ,データプレーンのみ透過的に高速化させる機構,shadowLBを設計,実装した.shadowLBは既存のコントロールプレーンのデファクトスタンダードであるipvsのAPIを用いながら,テールレイテンシにおいてX倍の性能を評価にて示した.





% 現在,IPv4を用いたインターネットからIPv6の新しいインターネットへの移行が加速している.アクセスネットワークを提供するISP(Internet Service Provider)の多くは,IPv4/IPv6デュアルスタック運用をおこなうことにより,すでにIPv6の利用環境整備を完了しており,PCやスマートフォンに代表されるクライアント端末ではIPv6の利用が前提となってきている.

% このような中,コンテンツ事業者やIDC(Internet Data Center)事業者の多くは,依然としてIPv4によるネットワーク運用・サービス提供を行っている.既に各地域レジストリの保有するIPv4アドレスプールは枯渇している上,新たなインターネット標準はIPv6に最適化した標準策定を行う方針が確認されているなど,IDCネットワークにおけるIPv4によるネットワーク運用は限界を迎えている.事業者が継続的にビジネスを拡大するためには,IPv4アドレスを極力使用しないIPv6シングルスタックネットワークの活用が不可欠である.

% しかしながらIPv4によるトラフィックは依然としてインターネット全体の大きな割合を占めており,IPv6シングルスタックネットワークであってもIPv4によるサービスを継続して提供可能なネットワーク設計が求められている.

% IPv6シングルスタックネットワークにおけるIPv4によるサービス提供技術として,SIIT-DCと呼ばれる手法がIETFにおいて標準化されている.SIIT-DCはBR(Border Relay)と呼ばれる変換機構をIPv4インターネットとIDC内のIPv6ネットワークの境界点に設置し,アドレス変換テーブル(EAMT:Explicit Address Mapping Table)を参照してプロトコル変換を行うことで,IPv6サーバでのIPv4サービス提供を可能にする技術である.

% しかしながらSIIT-DCではEAMTの動的な制御方法についての定義がなされておらず,複数のBRを運用する環境でのEAMTの一貫性の確保が難しい点や,IPv4でサービス提供を行なうサーバの構成変更が行われた際に個別の運用が必要になる点が課題に挙げられる.SIIT-DCのこれらの問題を解決するためには,BRが保有するEAMTを動的に制御する”ダイナミックEAMT機構”が必要である.

% 本研究ではBGP(Border Gateway Protocol)を利用したダイナミックEAMT機構を提案し,数多くのサービスを提供するIDC・コンテンツ事業者において,IPv6シングルスタック運用によるIPv4サービスの柔軟な提供を可能にした.BGPはインターネットで広く利用されている動的経路制御プロトコルであり,経路情報を一貫性のある形で共有することが出来るほか,BGPピアの変更に対して追従する機構を有している.本提案手法ではIBGP(Internal BGP)をEAMT管理機構に適応するための機能拡張を行うと共に,ルートリフレクタを利用したスケーラブルなダイナミックEAMT管理・制御機構を実現した.

% 本手法を評価するために,BGPによるEAMT管理制御機構を有するソフトウェアルータを実装し,IDCネットワークを模した概念実証用ネットワーク上で評価実験を行った.本評価実験により,本提案手法が1階層のルートリフレクタ構成を用いることで,最大600台のIPv4サービス提供サーバの情報から,30台のBRのEAMTを一貫性のある形でダイナミックに制御可能であることが明らかになった.また,多層のルートリフレクタ構成を用いることにより,さらに大規模な商用環境でもサービス提供を可能にするネットワーク設計も立案した.

% 本研究を通して,SIIT-DCのEAMTの一貫性の維持や変更追従性の面で,本提案手法が効果的に作用することが証明された.
% 本研究で提案した手法は,IPv6シングルスタックネットワークにおけるIPv4サービス提供品質の向上を実現し,今後のIDCネットワーク設計に貢献することが期待される.



~ \\

キーワード:\\
\underline{1. Load Balancer},
\underline{2. XDP},
\underline{3. 負荷分散}
\begin{flushright}
\dept \\
\author
\end{flushright}
