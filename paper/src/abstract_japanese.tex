修士論文要旨 - 2019年度 (令和元年度)
\begin{center}
\begin{large}
\begin{tabular}{|M{0.97\linewidth}|}
    \hline
      \title \\
    \hline
\end{tabular}
\end{large}
\end{center}

2020年現在,様々な理由からIDC(Internet Data Center)・コンテンツ事業者のIPv4を前提とした長期的なネットワーク運用は既に限界を迎えているが,IPv4によるトラフィックは依然としてインターネット全体の大きな割合を占めており,IPv6シングルスタックネットワークであってもIPv4によるサービスを継続して提供可能なネットワーク設計が求められている.

IPv6シングルスタックネットワークにおけるIPv4によるサービス提供法として,"SIIT-DC"と呼ばれる手法がIETFにおいて標準化されている.
SIIT-DCではBR(Border Relay)と呼ばれる変換機構をIPv4インターネットとIDC内のIPv6ネットワークの境界点に設置し,アドレス変換テーブル(EAMT:Eplicit Address Mapping Table)を参照してプロトコル変換を行うことで,IPv6サーバでのIPv4サービス提供を可能にする.しかしながらSIIT-DCではEAMTの動的な交換方法についての定義がなされておらず,EAMTが一貫しない場合にサービスの冗長性を維持出来ない点や,IPv4でサービス提供を行なうサーバの構成変更が行われた場合に個別の運用が必要になる点が課題に挙げられる.SIIT-DCのこれらの問題を解決するためには,システムによりBRが保有するEAMTを動的に制御する,"ダイナミックEAMT機構"が必要となる.


本研究ではBGP(Border Gateway Protocol)を利用したダイナミックEAMT機構を提案する.BGPはインターネットで広く利用されている動的経路制御プロトコルであり,経路情報を一貫性のある形で共有することが出来るほか,BGPピアの変更に対して追従する機構を有している.本提案手法ではIBGP(Internal BGP)及びルートリフレクタを利用し,スケーラビリティのあるダイナミックEAMT機構を実現する.

本手法を評価するために,新たにBGPによるアドレス変換テーブル制御機構を実装したソフトウェアルータを実装し,IDCネットワークを模した概念実証用ネットワーク上で評価実験を行った.
本評価環境においては,本提案手法が最大600台のIPv4サービス提供サーバの情報から,30台のBRのEAMTを一貫性のある形でダイナミックに制御可能なことが明らかになった.SIIT-DCのEAMTの一貫性の維持や変更追従性の面で,本提案手法が効果的に作用することが証明された.

本研究で提案した手法は,IPv6シングルスタックネットワークにおけるIPv4サービス提供の質向上を実現可能であり,今後のIDCネットワーク設計に貢献することが期待される.


~ \\

キーワード:\\
\underline{1. データセンターネットワーク},
\underline{2. ネットワークオペレーション},
\underline{3. IPv6移行技術}
\begin{flushright}
\dept \\
\author
\end{flushright}
