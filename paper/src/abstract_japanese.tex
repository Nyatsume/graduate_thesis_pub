修士論文要旨 - 2019年度 (令和元年度)
\begin{center}
\begin{large}
\begin{tabular}{|M{0.97\linewidth}|}
    \hline
      \title \\
    \hline
\end{tabular}
\end{large}
\end{center}

~ \\

2019年現在,IANAが保有するIPv4アドレスプールは既に枯渇しており,各地域レジストリからも2020年頃には新規割当が行えなくなることが予想されている.
一般に新規に IPv4アドレスの取得するためには民間取引市場を利用する方法が考えられるが,1アドレスあたりの単価は年々上昇しており,新規にIPv4ネットワークを構築するためのコストは日々上昇していくことが考えられる.IDC事業者・コンテンツ事業者がビジネスを拡大するためには,IPv4アドレスを極力使用しないIPv6シングルスタックネットワークの活用が不可欠になっている.

一方で2019年現在においてもIPv4によるアクセス・トラフィックは依然としてインターネット全体の大きな割合を占めていることから,IPv6シングルスタックネットワークでありながらIPv4によるサービスを継続して提供可能なネットワーク設計が必要になってくると言える.


IPv6のみ構築されたIPv6シングルスタックネットワークにおいて既存のIPv4クライアントに対してサービスを提供する方法として、ステートレスアドレス変換を利用した"SIIT-DC"と呼ばれるネットワークデザインがインターネット標準として標準化されている.SIIT-DCではBR(Border Relay)と呼ばれる変換ノードをIPv4ネットワーク・インターネットとの境界点ごとに設置し,明示的アドレス変換テーブル(EAMT: Eplicit Address Mapping Table)を参照してプロトコル変換を行い,IPv6ノードでのIPv4サービス提供を可能にする.
しかしながらSIIT-DCではEAMTの動的な交換方法についての定義がなされておらず,対外接続点が複数存在する場合の冗長性の維持が難しい点や,IPv4でサービス提供を行なうサーバの構成変更が行われた場合に運用負荷が非常に高くなる点が課題に挙げられる.

本研究ではBGPを利用したアドレス変換テーブルの広告・更新技術と,それを適切に運用するために必要なノード群の設計手法を提案する.これにより,SIIT-DCの課題であった冗長性の維持や構成変更へのに対して,ダイナミックに対応することが可能になる.

この手法を評価するために,新たにBGPによるアドレス変換テーブル制御機構を実装したソフトウェアルータを実装し,多くの対外接続点を持つ学術ISPであるWIDE Projectのバックボーンネットワークをモデルケースに,エミュレータを用いて概念検証実験を行った.考えられる他の手法と比較し,本手法が冗長性と柔軟性の点で優位であることが証明された.







~ \\
キーワード:\\
\underline{1. IPv6},
\underline{2. データセンターネットワーク},
\underline{3. ネットワークオペレーション},
\underline{4. IPv6移行技術}
\begin{flushright}
\dept \\
\author
\end{flushright}
