修士論文要旨 - 2019年度 (令和元年度)
\begin{center}
\begin{large}
\begin{tabular}{|M{0.97\linewidth}|}
    \hline
      \title \\
    \hline
\end{tabular}
\end{large}
\end{center}

% IDC市場の広がりIPアドレスの枯渇・IPv6への以降
2019年現在,ライブ映像配信のようなリアルタイムなサービスに対するニーズが年々高まっていることや,災害・地政学的リスクの回避を目的として,IDC(Internet Data Center)・コンテンツ事業者を中心にコンテンツ配信基盤の拠点を各地域に分散化する取り組みが活発である.

一方で2020年頃には各地域レジストリからIPv4アドレスの新規割当が行えなくなることが予想されている.加えて,2016年にIAB(Internet Architecture Board)によりインターネット標準はIPv6に最適化した標準策定を行う方針が確認されており,IPv4を前提とした長期的なネットワーク運用は限界を迎えていると言える.すなわち,IDC・コンテンツ事業者がビジネスを拡大するためには,IPv4アドレスを極力使用しないIPv6シングルスタックネットワークの活用が不可欠になっている.

しかしながら2019年現在においてもIPv4によるアクセス・トラフィックは依然としてインターネット全体の大きな割合を占めていることから,IPv6シングルスタックネットワークでありながらIPv4によるサービスを継続して提供可能なネットワーク設計が必要である.

IPv6のみ構築されたIPv6シングルスタックネットワークにおいて既存のIPv4クライアントに対してサービスを提供する方法として,"SIIT-DC"と呼ばれるネットワークデザインがインターネット標準化されている.SIIT-DCではBR(Border Relay)と呼ばれる変換機構をIPv4インターネットとIDC内のIPv6ネットワークの境界点に設置し,明示的アドレス変換テーブル(EAMT:Eplicit Address Mapping Table)を参照してプロトコル変換を行い,IPv6サーバでのIPv4サービス提供を可能にする.しかしながらSIIT-DCではEAMTの動的な交換方法についての定義がなされておらず,EAMTが一貫しない場合にサービスの冗長性を維持出来ない点や,IPv4でサービス提供を行なうサーバの構成変更が行われた場合に個別の運用が必要になる点が課題に挙げられる.

本研究ではBGP(Border Gateway Protocol)を利用したアドレス変換テーブルの広告・更新技術(ダイナミックEAMT)と,それを適切に運用するために必要なノード群の設計手法を提案する.これにより,SIIT-DCの課題であったサービスの冗長性や変更追従性に対して柔軟に対応するIPv6シングルスタックネットワークの構築が可能になる.

本手法を評価するために,新たにBGPによるアドレス変換テーブル制御機構を実装したソフトウェアルータを用いた評価実験用ネットワークを構築した.本評価環境を用いて概念検証実験を行い,EAMTの一貫性の維持や変更追従性及びスケーラビリティの面で,本手法が効果的に作用することが証明された.



~ \\
キーワード:\\
\underline{1. データセンターネットワーク},
\underline{3. ネットワークオペレーション},
\underline{4. IPv6移行技術}
\begin{flushright}
\dept \\
\author
\end{flushright}
