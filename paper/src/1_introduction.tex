\chapter{序論}
\label{introduction}
本章では本研究の背景とモチベーション,及び全体の構成について記述する.

\section{背景:IPv6シングルスタックネットワークに求められる役割}
\label{introduction:background}

\subsection{IDCネットワークを取り巻く環境}
\subsubsection{IDC市場の広がり}
ロバスト化・高品質化のために各地に拠点を広げる流れがある

\subsubsection{IPv4アドレスの枯渇}
2019年現在,IANAが保有するIPv4アドレスプールは既に枯渇しており,各RIRからも2020年頃には新規割当が行えなくなることが予想されている.一方で近年は民間事業者間アドレス取引も盛んに行われている.
一般に,新規にIPv4アドレスの割当を受けるためにはこのような民間取引市場を利用する方法が考えられるが,1アドレスあたりの単価は年々上昇しており,新規にIPv4ネットワークを構築するためのコストは日々上昇していくことが考えられる.

枯渇している図を書く

\subsection{IPv4/IPv6デュアルスタックネットワークの問題}
\label{introduction:background:dualstack_problems}
従来,アクセスネットワーク及びIDCネットワークではIPv6ネットワークの導入を行なう手法としてはIPv4/IPv6デュアルスタック構成を採用する手法が一般的であった.


IPv4不足の解決にはならない

OPEXの肥大化

機器の高性能化が必要


\subsection{IPv6シングルスタックネットワーク}
\label{introduction:background:IPv6-single-stack-network}
IDC事業者・コンテンツ事業者がビジネスを健全に拡大するためには,IPv4グローバルアドレスを極力使用しないIPv6ネットワークの活用が不可欠になっている.またネットワークスライシングやサービスチェイニングを実現する技術としてSRv6[6]の利用に注目が集まっており,柔軟でスケーラビリティのあるネットワークの導入を目的にしたIPv6ファブリックの構築機運が高まっている.

以下のような働きが期待される.
\subsubsection{シングルスタック採用によるOPEX/CAPEXの削減}
\subsubsection{IPv4サービスの提供}
\subsubsection{デュアルスタックネットワークと同等以上の性能}


\section{本研究の取り組み}
\subsection{モチベーション}
本研究では第\ref{introduction:background:IPv6-single-stack-network}項で述べたようなIPv6シングルスタックネットワークにおける諸問題のうち,IPv4サービスの提供における冗長性や構成変更への追従性を解決を目指す.

\subsection{取り組みの概要}
SIIT-DC + BGPによるダイナミックEAMTで実現する



\section{本論文の構成}

本論文における以降の構成は次の通りである.

~\ref{background}章では,背景を述べる.
~\ref{issue}章では,本研究における問題の定義と,解決するための要件の整理を行う.
~\ref{proposed}章では,本研究の提案手法を述べる.
~\ref{implementation}章では,~\ref{proposed}章で述べたシステムの実装について述べる.
~\ref{evaluation}章では,\ref{issue}章で求められた課題に対しての評価を行い,考察する.
~\ref{conclusion}章では,本研究のまとめと今後の課題についてまとめる.


%%% Local Variables:
%%% mode: japanese-latex
%%% TeX-master: "../thesis"
%%% End:
