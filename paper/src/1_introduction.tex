\chapter{序論}
\label{introduction}

本章では本研究の背景,課題及び手法を提示し,本研究の概要を示す.

\section{はじめに}
\label{introduction:background}

2019 年現在,IANA*3が保有する IPv4 アドレスプールは既に枯渇しており,各 RIR*4からも2020年頃には新規割当が行えなくなることが予想されている. 一方で近年は民間事業者間アドレス取引も盛んに行われている.
一般に,新規に IPv4 アドレスの割当を受けるためにはこのような民間取引市場を利用する方法が考えられるが,1アドレスあたりの単価は年々上昇しており,新規にIPv4 ネットワークを構築するためのコストは日々上昇していくことが考えられる.

即ち IDC 事業者・コンテンツ事業者がビジネスを拡大するためには,IPv4 グローバルアドレスを極力使用しないIPv6 ネットワークの活用が不可欠になっている.またネットワークスライシングやサービスチェイニングを実現する技術としてSRv6[6]の利用に注目が集まっており,柔軟でスケーラビリティのあるネットワークの導入を目的にしたIPv6ファブリックの構築機運が高まっている.


\section{本論文の構成}

本論文における以降の構成は次の通りである.

~\ref{background}章では,背景を述べる.
~\ref{issue}章では,本研究における問題の定義と,解決するための要件の整理を行う.
~\ref{proposed}章では,本研究の提案手法を述べる.
~\ref{implementation}章では,~\ref{proposed}章で述べたシステムの実装について述べる.
~\ref{evaluation}章では,\ref{issue}章で求められた課題に対しての評価を行い,考察する.
~\ref{conclusion}章では,本研究のまとめと今後の課題についてまとめる.


%%% Local Variables:
%%% mode: japanese-latex
%%% TeX-master: "../thesis"
%%% End:
