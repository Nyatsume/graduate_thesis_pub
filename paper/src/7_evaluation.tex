\chapter{評価}
\label{evaluation}
本章では,第\ref{proposal}章及び第\ref{implementation}章で設計・実装に関して述べた本提案手法に関して,第\ref{issue:siit-dc_problems}節で指摘したSIIT-DCの課題に対して有効性があることに対して定量的に評価を行う.



\section{評価要件}
SIIT-DCにおけるDynamicEAMT機構では,第\ref{consideration:points}節で述べた事項が求められる.

それらを定量的に評価するための指標・尺度について記述する.


\subsection{BR間のEAMTの一貫性}
SIIT-DCネットワークにおいて,各BRが一貫したEAMTを持つまでの時間を調べる.数を増やした場合に線形に変化するかどうか

\subsubsection{変更追従性}
サーバー構成が変更になった場合に,どれだけの時間でサービス提供が行えるかを計測する.

\subsubsection{スケーラビリティ}
表\ref{table:compare_approach}で示した定性評価により,本提案手法にスケーラビリティがあることがわかる.


\section{実験環境}
第\label{implementation:poc}項で記述したPoC実装を利用する.

ネットワークエミュレータ環境として,GNS3を利用する.



\section{実験シナリオ1: SIIT-DCネットワークの構築}
このシナリオにおける,BRの数, サーバーの数を変数とした時に変化を調べる.

\subsection{ネットワーク構成}
図を張る. BRとサーバーは水平スケールさせる.


\subsection{実験結果}
EAMTが一貫するまでの時間を調べる.

\subsection{考察}
一貫性保持が成功.

線形に変化しており,O(n)の十分なスケーラビリティがある.


\section{実験シナリオ2: サーバーの構成変更}
このシナリオにおける,BRの数, サーバーの数を変数とした時に変化を調べる.

サーバーの削除からのFailover(LPで重み替え)



\subsection{ネットワーク構成}
図を張る. BRとサーバーは水平スケールさせる.

\subsection{実験結果}
追加したサーバー1台がサービス開始出来るまでの時間を,サーバー台数とBR台数が増えていった場合にも変わらず出来ることを証明.


\subsection{考察}
リニアに追従.

線形に変化しており,O(n)の十分なスケーラビリティがある.


%%% Local Variables:
%%% mode: japanese-latex
%%% TeX-master: "../thesis"
%%% End:
