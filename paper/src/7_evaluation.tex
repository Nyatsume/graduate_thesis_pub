\chapter{評価}
\label{evaluation}
本章では,第\ref{proposal}章及び第\ref{implementation}章で設計・実装に関して述べた本提案手法に関して,第\ref{issue:siit-dc_problems}節で指摘したSIIT-DCの課題に対して有効性があることに対して定量的に評価する.



\section{評価要件}
SIIT-DCにおけるDynamicEAMT機構では,第\ref{consideration:points}節で述べた事項が求められる.
それらを定量的に評価するための指標・尺度について記述する.


\subsection{BR間のEAMTの一貫性}
第\ref{issue:siit-dc_problems:consistent}項で示したように,SIIT-DCネットワークにおける各BRは一貫性のあるEAMTを保持する必要がある.
本評価実験では後に述べるテストケースにおいて,以下の2点を検証することで各BRのEAMTが一貫性を有すると定義する.
\begin{itemize}
    \item \textbf{EAM数の収束} \\
    EAMの更新が行われる事象が発生した場合に,一定の収束時間が経過した後に各BRのEAMTのレコード数が変化することなく,一意に収束すること.
    \item \textbf{EAMTのレコードの内容} \\
    EAM数が収束した際に各BRの有するEAMTの内容に差異が無いこと.
\end{itemize}

\subsection{変更追従性}
第\ref{issue:siit-dc_problems:follow}で指摘したように,従来のSIIT-DCの構成ではIPv4サービスが追加・削除・変更された場合やBRの追加配備を行った場合に,各BRのEAMTの内容を適宜変更する必要があった.
%本評価実験では下記のN点の要素を満たすことを,EAMTがIPv4サービスの構成変更に動的に追従する状態であると定める.

本研究ではEAMT変更が必要な事象が発生した場合に,運用者が特別なオペレーションを行うことなくサービス提供を想定通りに開始・継続・中断される状況を,EAMTが変更に追従している状態であると定義する.

本評価実験では下記の様なEAMT変更が必要な事象を想定する.

\begin{itemize}
    \item \textbf{IPv4サービスの追加・削除} \\
    ネットワーク内のIPv6サービスにおいて,IPv4サービスアドレスを付与することでIPv4/IPv6の両プロトコルによるサービス提供を開始する場合.もしくは,そのサーバーが行っているIPv4サービス提供を中断する場合.
    \item \textbf{IPv4サービス提供サーバーの変更} \\
    SIIT-DCにおいて提供中のIPv4サービスに関して,他のIPv6アドレスを持つサーバーによるサービス提供を改めて開始する場合.
    \item \textbf{BRの追加・撤去} \\ 
    対外接続点に新しいBRを追加もしくは稼働中のBRを撤去しながらIPv4サービスを継続して行う場合.
\end{itemize}

% AWSのElastic LoadbalancerのTimeout時間は5秒. 1サービスアドレうあたり5秒以内に収束したら良いのでは.

\subsection{スケーラビリティ}
第\ref{consideration:points:scalability}で述べたように,DynamicEAMTを実現する機構は対外接続点やIPv4提供サービスの増加に柔軟に対応可能であることが望ましい.

本評価実験では各ノードのスケールにおいては下記を想定し,IPv4提供サービスの総数が増加した場合においても,オーダーが増加することなく動作が可能であることを評価する.

\begin{itemize}
    \item BR: 2-30ホスト 
    \item RR: 1-4ホスト 
    \item IPv4提供サーバー: 120ホスト
    \item EAM数: 0 - 10000
\end{itemize}




\section{想定するネットワークトポロジー}
第\ref{issue:siit-dc-merit}節で述べたように,SIIT-DCは様々なネットワークトポロジーでの活用が可能である.


\section{実験環境}
第\label{implementation:poc}項で記述したPoC実装を利用する.


\section{実験シナリオ1: SIIT-DCネットワークの構築}
このシナリオにおける,BRの数, サーバーの数を変数とした時に変化を調べる.

\subsection{ネットワーク構成}
図を張る. BRとサーバーは水平スケールさせる.


\subsection{実験結果}
EAMTが一貫するまでの時間を調べる.

\subsection{考察}
一貫性保持が成功.

線形に変化しており,O(n)の十分なスケーラビリティがある.


\section{実験シナリオ2: サーバーの構成変更}
このシナリオにおける,BRの数, サーバーの数を変数とした時に変化を調べる.

サーバーの削除からのFailover(LPで重み替え)



\subsection{ネットワーク構成}
図を張る. BRとサーバーは水平スケールさせる.

\subsection{実験結果}
追加したサーバー1台がサービス開始出来るまでの時間を,サーバー台数とBR台数が増えていった場合にも変わらず出来ることを証明.


\subsection{考察}
リニアに追従.

線形に変化しており,O(n)の十分なスケーラビリティがある.


%%% Local Variables:
%%% mode: japanese-latex
%%% TeX-master: "../thesis"
%%% End:
