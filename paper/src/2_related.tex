\chapter{IPv6シングルスタックネットワークでのIPv4サービス提供手法}
\label{related}
本章ではIPv6シングルスタックネットワークでのIPv4サービス提供手法を比較し,検討する.

\section{IPv4 as a Service}

目的: IPv4 as a Service
 IPv6 Single Stackネットワークにおける透過的なIPv4サービス提供
IPv4 aa S 手法分類
https://www.a10networks.co.jp/products/features/IPv4.html

 大別して以下の3種に分けられる。
Proxy 手法
例:LBによるプロキシ
IPv4⇔IPv6 SLB
Tunne手法
例:
ネットワークプロトコル変換法
例:SIIT-DC
各手法の比較
星取り表で比較

 プロトコルの柔軟さ

いろんなプロトコルで利用できるかなど
事業者の規模的にProxy法は複数事業者で運営しにくい。
 デプロイメントの容易さ

どれだけの機器を入れなくちゃいけないとか

中継機器で考えなくちゃいけないことなど

 IPv4 アドレスの節約度

どれだけ実際にv4アドレスを節約出来るかなど

まとめ
 ネットワーク・プロトコル変換法が良いアルと高らかに宣言